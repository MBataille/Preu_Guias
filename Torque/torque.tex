\documentclass[letterpaper]{article}
\usepackage[utf8]{inputenc}
\usepackage[spanish]{babel}
\usepackage[letterpaper,includeheadfoot, top=.5cm, bottom=3.0cm, right=2.0cm, left=2.0cm]{geometry}
\renewcommand{\familydefault}{\sfdefault}
\usepackage{amsmath}
\usepackage{graphicx}
\usepackage{subcaption}
\usepackage{gensymb}
\usepackage{color}
\usepackage{hyperref}
\usepackage{amssymb}
\usepackage{url}
%\usepackage{pdfpages}
\usepackage{fancyhdr}
\usepackage{enumerate}
\usepackage{float}
\usepackage{tikz}
\usetikzlibrary{patterns}
\usepackage{siunitx}
\usepackage{framed}
\tikzset{
every picture/.append style={
  execute at begin picture={\deactivatequoting},
  execute at end picture={\activatequoting}
  }
}
%-------------------- CABECERA ---------------------
\pagestyle{fancy}
\fancyhf{}
\author{Martin Bataille}
\date{}
\title{\bf Torque}
%Encabezado
\fancyhead[R]{\includegraphics[scale=0.35]{jct.jpg}}
\fancyfoot[C]{\thepage}


\newcommand{\tpb}[1]{node[midway, below, sloped] {#1}}
\newcommand{\tpa}[1]{node[midway, above, sloped] {#1}}
\newcommand{\tvec}[3]{[->, thick] #1 -- #2 \tpb{#3}}
\newcommand{\tveca}[3]{[->, thick] #1 -- #2 \tpa{#3}}
\newcommand{\tvecnotsloped}[3]{[->, thick] #1 -- #2 {node[midway, above] {#3}}}

\newcounter{propiedades}
\newcounter{definiciones}

\newcommand{\propi}{\stepcounter{propiedades} \textbf{Propiedad \thepropiedades}: }
\newcommand{\defii}{\stepcounter{definiciones} \textbf{Definición \thedefiniciones}: }

\newenvironment{prop}
{ \begin{framed} \propi}
{ \end{framed} }
\newenvironment{defi}{\begin{framed} \defii}{\end{framed}}

\renewcommand{\sectionmark}[1]{\markright{\thesection.\ #1}}
\renewcommand{\headrulewidth}{0.5pt}
\renewcommand{\footrulewidth}{0pt}
\setlength{\headheight}{92pt}

% --------------- ---------PORTADA -----------------------
\begin{document}
\maketitle
\thispagestyle{fancy}
\begin{center}
\includegraphics[scale=0.6]{portada.jpg}
\end{center}
\pagebreak

¿Por qué el pomo de la puerta está del lado opuesto a las bisagras? ¿Qué cambiaría si uno pone el pomo de la puerta del lado de las bisagras? ¿Por qué no se puede balancear un lápiz sobre su punta? Aunque estas preguntas puedan parecer muy estúpidas, la respuesta no es trivial. Todo esto y mucho más lo resolveremos al final de esta guía!

\section*{Introducción}

Hemos aprendido que las fuerzas provocan cambios en el movimiento de los objetos a través de $F = ma$. Sin embargo, esto solo explica la traslación de los objetos que constituye solo una parte del movimiento de los cuerpos. Un ejemplo se presenta cuando lanzamos una botella plástica esperando que caiga de verticalmente y quede estable, procediendo después a hacer un \emph{dab}. Ahí la botella no solo se traslada de nuestra mano hasta el piso (de lo contrario sería demasiado fácil y no se merecería un \emph{dab}), sino que también rota en el aire. Pero, ¿por qué se produce esa rotación?

Como veremos a lo largo de esta guía, esa rotación se debe a que ejercemos una fuerza sobre el cuerpo, esa fuerza provoca un torque en el cuerpo y el torque se traduce en movimiento rotacional.

\section*{¿Cómo se define el torque?}

Antes de definir el torque tenemos que entender unos conceptos primero. Al momento de abrir una puerta, uno puede comprobar experimentalmente dos cosas: 
\begin{itemize}

\item es más fácil abrirla mientras más lejos del eje de rotación (donde están las bisagras) la empujemos.

\item es más fácil abrirla mientras el ángulo entre la dirección de la fuerza y la puerta se acerca a \ang{90}.

\end{itemize}

Podemos entonces deducir que antes de definir el torque tenemos que definir esta distancia entre el eje de rotación y el punto donde se aplica la fuerza, y el ángulo. Para esto es muy útil pensar en el vector que va desde el eje de rotación hasta el punto donde se aplica la fuerza. A este vector se le llama brazo de palanca.

\begin{defi}
El brazo de palanca se suele escribir como $\vec{r}$ o $\vec{b}$ y representa el vector que va desde el eje de rotación hasta el punto donde se aplica la fuerza.
\end{defi}

Volviendo al ejemplo de la puerta, podemos pensar en el caso que aplicamos dos fuerzas en distintos puntos y con distinta dirección, como en la figura siguiente. El problema es, ¿qué fuerza permite abrir con mayor fácilidad la puerta? La respuesta está en el torque.

\begin{figure}[h]
\centering
\begin{tikzpicture}
\fill [pattern=north east lines] (0,0) rectangle (5,0.5);
\draw (0,0) rectangle (5,0.5);
\draw [->] (4.5,0) -- (6,2) node[midway, below, right] {$\vec{F}_2$};
%\draw (4.5,2.5) node[above] {$\vec{F}_2$};
\draw [->] (0,-0.6) -- (4.5,-0.6) node[midway, below] {$\vec{r}_2$}; 
\draw [->] (3.5,0) -- (3.5,2.5);
\draw (3.5,2.5) node[above] {$\vec{F}_1$};
\draw [->] (0, -0.1) -- (3.5, -0.1) node[midway, below] {$\vec{r}_1$};
\end{tikzpicture}
\end{figure}

\begin{defi}
El torque o momento de fuerza es un vector que se define como el producto cruz entre el brazo de palanca $\vec{r}$ y la fuerza aplicada $\vec{F}$,
$$\vec{\tau} = \vec{r}\times\vec{F}$$

Lo que es equivalente a,
$$|\vec{\tau}| = |\vec{r}|\cdot|\vec{F}|\cdot\cos{\theta}$$

Donde $\theta$ representa el ángulo que se forma entre la fuerza aplicada y el brazo de palanca, y la dirección de $\vec{\tau}$ está dada por la regla de la mano derecha.

Se deduce que la magnitud del torque se mide en \si{N.m}
\end{defi}

Volviendo al ejemplo de la puerta, tomemos $r_1 = 0.4\ \si{m}$, $F_1 = 10\ \si{N}$, $\theta_1 = \frac\pi2\ \si{rad}$, $r_2 = 0.8\ \si{m}$, $F_2 = 10\ \si{N}$ y $\theta_2 = \frac\pi3\ \si{rad}$, donde $\theta_2$ representa el ángulo entre $\vec{r}_2$ y $\vec{F}_2$. ¿Qué fuerza ejerce un mayor torque?

Primero, notemos que por la regla de la mano derecha, $\vec{\tau}_1$ y $\vec{\tau}_2$ tienen la misma dirección y sentido, basta entonces ver qué vector tiene mayor magnitud. Luego, aplicamos la definición de torque:
\begin{align*}
|\vec{\tau}_1| &= |\vec{r}_1|\cdot|\vec{F}_1|\cdot\cos{\theta_1} = 0.4 \cdot 10 \cdot \cos{\frac\pi2} = 4\ \si{N.m} \\
|\vec{\tau}_2| &= |\vec{r}_2|\cdot|\vec{F}_2|\cdot\cos{\theta_2} = 0.8 \cdot 10 \cdot \cos{\frac\pi3} = 8\cdot\frac12 = 4\ \si{N.m}
\end{align*}





\section*{Ejercicios}

\begin{enumerate}

\item Pendiente

\end{enumerate}

\section*{Problemas}

\begin{enumerate}

\item Pendiente

\end{enumerate}

\end{document}